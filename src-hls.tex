%\nocite{0d7b89ad}
%\nocite{66edf75b}
%\nocite{ea5d49bf}
%\nocite{0349e8ea}
%\nocite{2d5c2e63}
%\nocite{e5194763}
%\nocite{00000011}
For the record: in this subsection let
\begin{align*}
  n,
  n_{1},
  n_{2}
  \in
  \mathbb{N}
\end{align*}
while
\begin{align*}
  N,
  N_{1},
  N_{2}
  \in
  \mathbb{N}
  \cup
  \lbrace
    \infty
  \rbrace
\end{align*}
Furthermore, whenever we talk about $N$-groupoids and $N$-categories we mean weak $N$-groupoids and $N$-categories, respectively.\footnote{note that $0$-groupoids and $0$-categories are just mere sets} And if we do not mean weak we say strict. This is a common convention. Last let us agree to say $n$-morphism both for higher groupoids and higher categories while if we have to restrict to one of these cases we say $n$-path in the higher groupoid case and $n$-arrow in the higher category case. This is just a convenient convention here without too much intended meaning.
\\\\
Higher-Level structures are informally understood as structures on higher-level sets analogously as structures on sets we informally introduced in remark \ref{rem:c3trick}. The questions are:
\begin{enumerate}
\item[$\bullet$]
What is a higher-level set?
\item[$\bullet$]
What do higher-level sets have to do with category theory?
\end{enumerate}
Well, there are different ideas what higher-level sets can be. We take the stance here, that a next-level set should be a set such that any two elements of this set are linked by arbitrarily many e.g. paths - directed or undirected - that can be concatentated in aribtrary ways where one has to respect direction in case of directed paths, of course. More concisely, a next-level set should be a set plus a second set containing e.g. paths (or whatever) structuring the first set. And we end up with two sets - or better say a $2$-dimensional set. But be aware that we are not talking about ordinary undirected and directed graphs. What we are talking about are nothing but pre-formal groupoid theory and category theory\footnote{categories in TG as we defined them are equivalent to something that in graph theory goes by the name \textit{directed multidigraph}}. As we have learned in subsection \ref{sec:nt} we can structure a set of $1$-morphisms by $2$-morphisms in the same way we structured a set (of $0$-morphisms) by $1$-morphisms between elements to yield a weak $2$-groupoid and weak $2$-category, respectively. In other words a $3$-dimensional set. Carrying on recursively in this manner we end up with $n+1$-dimensional set as either a weak $n$-groupoid or weak $n$-category. The modern idea is then to replace sets as the only mathematical objects by weak $\infty$-groupoids or weak $\infty$-categories (or any other infinity version of higher-level set you can think of) in a mathematical theory of everthing.
\\
So now what? $n$-paths or $n$-arrows? There is a guy (\cite{e5194763}) opposing the idea of pre-formal $n$-arrows here by arguing that undirected paths are more fundamental than directed ones in the same way as manifolds are more fundamental than oriented manifolds, that is, a directed path in our sense would be just an undirected path with an additional structure of direction. However, he considers categories as a possibility of next-level sets albeit in a different manner as we do here. Yet we can also argue for directed paths instead of undirected paths here since one can consider directed paths as equally (or even more) fundamental as undirected paths for the following reason: one can intuitively think of directed paths described by category theory as paths with slope which can never be gone uphill. This is a synthetic description of directed paths which contain ordinary (undirected) paths as the special case of horizontal paths. To make this fully work geometrically for a arrow $f_{12} \colon X_{1} \rightarrow X_{2}$ we should draw objects $X_{1},X_{2}$ as vertical straight lines and arrows $f_{12}$ as areas having criss-cross lines where we are never allowed to follow these uphill. So if the arrow $f_{12}$ is an isomorphism we would draw
\[
\begin{tikzpicture}[scale=0.75]
  \filldraw[draw=white,pattern=horizontal lines]
    (-1,1)
    rectangle
    (1,5);
  \draw
    (-1,1)
    --
    (-1,5);
  \draw
    (1,1)
    --
    (1,5);
  \draw[->]
    (-1,0)
    -- node[above] {$f_{12}$}
    (1,0);
  \draw
    (-1,0) node[left] {$X_{1}$}
    --
    (-1,0);
  \draw
    (1,0)
    --
    (1,0) node[right] {$X_{2}$};
\end{tikzpicture}
\]
and else
\[
\begin{tikzpicture}[scale=0.75]
  \filldraw[draw=white,pattern=north west lines]
    (-1,1)
    rectangle
    (1,5);
  \draw
    (-1,1)
    --
    (-1,5);
  \draw
    (1,1)
    --
    (1,5);
  \draw[->]
    (-1,0)
    -- node[above] {$f_{12}$}
    (1,0);
  \draw
    (-1,0) node[left] {$X_{1}$}
    --
    (-1,0);
  \draw
    (1,0)
    --
    (1,0) node[right] {$X_{2}$};
\end{tikzpicture}
\]
hence a groupoid would be the special case when all arrows are illustrated by areas filled with horizontal paths. And, in fact, the guy in \cite{e5194763} concedes that philosophically $1$-arrows as in our approach to category theory are also legit as basic mathematical objects and do not necessarily have to be considered $1$-paths with additional structure. He even admits his line of argument as {\glqq}moot{\grqq} if somebody finds something such as UFP-HoTT for directed paths. That is, a formal synthetic theory of so-called {\glqq}directed homotopy theory{\grqq} - (synthetic) homotopy theory with directed $n$-paths, or $n$-arrows following our terminiological convention - as UFP-HoTT is (synthetic) homotopy theory with ordinary (undirected) $n$-paths. But at the moment UFP-HoTT is far more elaborated\footnote{indeed, it is mature enough to serve as a foundation of mathematics with basic mathematical objects $\infty$-groupoids} than any directed version of it. Which is another argument in \cite{e5194763} which we recommend as lazy weekend literature for the interested reader. By the way, the same author has written some quite worthwhile notes \cite{2d5c2e63} about intuition for homotopy type theory in a similar style we also want to recommend w.r.t. this subject.
\\\\
Now whatever we choose as higher-level set - higher groupoids or higher categories - we deal with weak versions of them and this suggests that equality should always be equivalence when talking about $N$-dimensional sets. This means (strict) equality on the $N$-th level and coherent isomorphism else unless $N = \infty$ where it just means coherent isomorphism. For sets this is equality and for $2$-dimensional sets it is isomorphism on $0$-morphisms and equality on $1$-morphisms. Note that if there was an $n+1$-st level then strict equality on the $n$-th level is identity. UFP-HoTT has a pretty nice perspective on this: identity isomorphisms are no better than coherent isomorphism, one just has to keep track which coherent isomorphism one took to prove equivalence. This is univalence and expresses that (strict) equality is expanded to equivalence. The modern approach to avoid this asymmetry of having both equality and isomorphisms in the categorical context is to consider $(N_{1},N_{2})$-categories which have arrows on all levels but all $k$-arrows with $k > N_{1} \geq N_{2}$ are identities and all $k$-arrows with $N_{1} \geq k > N_{2}$ are invertible. Equivalence is then just coherent isomorphism. In this setting $N$-categories are $(N,N)$-categories. $(N_{1},N_{2})$-categories are hybrids of $\infty$-groupoids and $\infty$-categories and might be used to find an $\infty$-category generalization of UFP-HoTT, where progress is made in the $(\infty,1)$-category case. $(\infty,1)$-categories are well-understood. You can imagine these as ordinary categories with the morphism sets not just $1$-dimensional sets but $\infty$-groupoids which is to say homotopy types by the Grothendieck hypothesis \ref{prp:groth}. This is needed when doing homotopy theory.\footnote{just think of fiber sequences in $\mathbf{Top}$, for example, where one has to topologize some morphism sets to make things work out} So $(\infty,1)$-categories provide a basic setting for doing homotopy theory. But we cannot expect this to suffice just like we couldn't expect set theory to work in any category equally well. There we found topos theory which provides a setting to do reasonable set theory. Therefore we would expect that a sensible notion of $(\infty,1)$-topoi is a setting for homotopy theory in that the objects of an $(\infty,1)$-topoi behave like homotopy types. This is, in fact, the case. However, we cannot further elaborate on that since we lack model categories as a $(1,1)$-categorical formulation of homotopy theory on the one hand and a formal definition of $(\infty,1)$-categories on the other. Moreover we lack some categorical ideas from later sections - fibrations (section \ref{sec:fibration}) and simplical sets (section \ref{sec:sset})- which are usually utilized in the formal definitions we would need. But let us already at this point refer to \cite{0349e8ea} to formally understand these things. In particular, the $(\infty,1)$-topos theory as the title of \cite{0349e8ea} suggests. \cite{0349e8ea} seems to be the undisputed standard reference on $(\infty,1)$-topoi theory. The internal language of $(\infty,1)$-topoi seems to be a homotopy type theory. Conclusively let us say that an $(\infty,1)$-topos is a place where one can do homotopy theory.
\\n
Where we are at Lurie's work anyways, a historical remark: one of the main reasons to prefer $(N,N)$-categories over weak $N$-categories is that there are problems proving the so-called cobordism hypothesis due to Baez for the latter but Lurie proved it in case of $(\infty,N)$-categories. For more on this see \cite{ea5d49bf} and \cite{00000011}.
\\\\
After we have an idea what higher-level sets should be we turn to the question of how we can transfer structures on $N_{1}$-dimensional sets to according structures on $N_{2}$-dimensional sets.\footnote{a $0$-dimensional set can be seen as the truth values {\glqq}true{\grqq} and {\glqq}false{\grqq}} In particular, if we are given a method to transfer some sorts of structures on $N_{1}$-dimensional sets to according structures on $N_{2}$-dimensional sets then if the method is applied to a mere\footnote{without structure} $N_{1}$-dimensional set it should yield an $N_{2}$-dimensional set. This is a minimal condition for any sensible method and if it holds in the case $n_{2} = n_{1} + 1$ for all $n_{1},n_{2}$ we can construct - and hence define - $n$-dimensional sets for all $n$ by such a method.
\\
However this is not applicable directly to get $\infty$-dimensional sets and one has to rely on other methods. To get $\infty$-groupoids in UFP-HoTT, for example, the types are characterized by deductive steps which make all types behave like $\infty$-groupoids - or more intuitively: spaces. $n$-groupoids or better say $n$-types are then defined as the $\infty$-groupoids with only identities above level $n$. This allows to construct the $n+1$-groupoid/type of $n$-groupoids/types of some (univalent) universe\footnote{the idea is the same as for Grothendieck universes though UFP-HoTT uses Russell-style universes} $\mathcal{U}$ of types/$\infty$-groupoids as a subtype by a so-called dependent sum\footnote{something like a collection of ordered pairs where the second coordinate depends on the first} with first coordinate an inhabitant of the universe and the second a proof that this inhabitant is an $n$-groupoid/type. As a formula this looks like
\begin{align*}
  n\textrm{-Type}
  \doteq
  n\textrm{-Type}_{\mathcal{U}}
  &:=
  \sum_{T \colon \mathcal{U}}
  \textrm{is-}n\textrm{-type}(T)
\end{align*}
Let us agree that $\infty\textrm{-Type}_{\mathcal{U}}$ is just $\mathcal{U}$. Moreover there are rules for how to make the higher-groupoid structure of any $\infty$-groupoid trivial above some level $n$ to yield an $n$-groupoid. This is called $n$-truncation and corresponds to the $n$-th Postnikov section in classical homotopty theory with ordinary topological spaces. But unlike Postnikov towers in the classical sense $n$-truncation is formally built-in into UFP-HoTT. This is an important feature of UFP-HoTT. We could now hope to transfer higher-level structures by $n$-truncation. Namely assume a type $T$ and an $n$-type $T_{n}$ including some known structure, that is, the structure as part of the types. Then if $T_{n}$ is the $n$-truncation of $T$ and if the $n$-truncation has a right inverse - or better say is a section - then we would consider the type $T$ as a type with structure of the same sort as the structure on $T_{n}$. Note that $T$ can be a groupoid of any dimension and is not necessarily a non-trivial $\infty$-groupoid. But usually the dimension of $T$ is higher than $n$. We do not want to dig deeper at this point but rather turn to a process similar in spirit which also works very well. One might call it weak internalization:
\begin{description}
\item[Step 1]
define a structure on an inhabitant of $N\textrm{-Type}$ as something internal to $N\textrm{-Type}$ for some $N$
\item[Step 2]
try to use this internal definition for any $N$
\end{description}
For example one can define a (weak) $\infty$-group to be a (weak) group object of $\mathcal{U}$, then a weak $N$-group is a (weak) group object of the type of $N\textrm{-Type}_{\mathcal{U}}$. If you do not yet understand this you will after reading UFP-HoTT. But it is not so important for what follows. 
\\
One can think of similar ideas for $\infty$-categories if we have a serious definition of them at all.\footnote{at best a generalization of UFP-HoTT in the sense of a synthetic theory of directed homotopy theory} One can then again think of something like the $(n+1,n+1)$-category of $(n,n)$-categories ${}_{(n,n)}\mathbf{Cat}_{\mathcal{U}}$ where we agree to regard ${}_{(\infty,\infty)}\mathbf{Cat}_{\mathcal{U}}$ as a universe of $\infty$-categories. We can then hope to find an $n$-truncation which truncates an $\infty$-category to an $(n,n)$-category by making all arrows above level $n$-trivial. Hence we can hope to transfer higher-level structures by $n$-truncation corresponding to the higher groupoid case. Note that by $(\infty,n)$-categories we have a hybrid of $\infty$-groupoids and $\infty$-categories which suggests a weaker form of $n$-truncation such that an $\infty$-category is $n$-truncated to a an $(\infty,n)$-category by discarding all non-invertible arrows above level $n$. However, this shall not concern us here since the stronger notion is the one which corresponds to $n$-truncation in UFP-HoTT. And transferring higher-level structures by this strong $n$-truncation applies. In this way one can for example tranfer the structure of natural numbers on a set to the $1$-categorical context yielding $\mathbf{Finset}$. On the other hand, it is straight forward to generalize the idea of weak internalization as:
\begin{description}
\item[Step 1]
define a structure on an $N$-category as something internal to ${}_{(n,n)}\mathbf{Cat}$
\item[Step 2]
try to use this internal definition for any $N$
\end{description}
In subsection \ref{sec:internaliz} we got a taste of the importance of internalizaztion. So it is no surprise that weak internalization is at least as important: it is as least as good as internalization since it is internalization in the $(\infty,1)$-categorical context which is a generalization of category theory.
\\\\
If we are honest, the most likely case one encounters in practice is that one is given a structure on a set and wants to make sense of it in the $2$-dimensional set case. At best one hopes that this process is recursive and translates to a process to go from $n$ to $n+1$. Hence we want to generalize a structure on a set to a structure on a category or more general from the $n$ case to the $n+1$ case by an as of yet mostly informal algorithm which tells us how to do so. In the following we discuss two such ideas: (recursive) categorification\footnote{some people call it vertical categorification} and (recursive) oidification\footnote{some people call it horizontal categorification}.\footnote{the word recursive is not part of the standard terminology but we find it less confusing at this point since transferring higher-level structures by $n$-truncation and weak internalization are also considered a kind of categorification where we avoid the terminology, too}
\\
But let us first reconsider how the dimension terminology of higher-level sets emerges if we want $n+1$-dimensional sets to be $n$-categories. What we mean is that a set is just an unstructured collection but a category $\mathbf{C}$ consists essentially of a set of objects $\mathrm{ob}_{\mathbf{C}}$ and a set of arrows $\mathrm{Mor}_{\mathbf{C}}$ (governed by composition) - a $2$-tuple - while a $3$-category consists of a set of $0$-arrows (objects), a set of $1$-arrows (morphisms) and a set of $2$-arrows (governed by two kinds of composition: vertical and horizontal) - a $3$-tuple. This process extends to an arbitrary $n$ (provided we have a definition of $n$-category). It is these $n$-tuples we mean when we say set of dimension $n$. Moreover, note that a functor is a $2$-dimensional function in this sense since it essentially consists of two functions - one on objects and one on morphisms. And an $N$-functor between $N$-categories should be just like functions $f_{n^{\backprime}}$ where $n^{\backprime} \in \mathbb{N}_{N}$ such that $f_{n^{\backprime}}$ maps $n^{\backprime}$-morphism to $n^{\backprime}$-morphisms respecting the various compositions and identities in a weak way, that is, up to equuivalence.\footnote{one also gets coherence conditions for $n$-functors}. Moreover note that a natural transformation is a function mapping $0$-morphisms to $1$-morphisms respecting the functor application.\footnote{it doesn't matter if we first apply the functor and then transform or the other way around} In the $N$-category case this suggests to consider $N$-functors and $N$-natural transformations as special cases of a more general notion called transfor\footnote{made-up word from functor and natural transformation}. Let $n_{0} \in \mathbb{N}_{N}^{\times}$ then if $(N,n_{0}-1)$-transfor is defined, an $(N,n_{0})$-transfor should be a family of functions indexed by $n^{\backprime} \leq N - n_{0}$ mapping $n^{\backprime}$-morphisms to $n^{\backprime}+n_{0}$-morphisms respecting $(N,n_{0}-1)$-transfor application up to equivalence where the $(N,0)$-transfor case should be just an $N$-functor. $(N,1)$-transfors shall then be $N$-natural transformations. Let us illustrate this a bit for $N \neq \infty$. If
\[
\begin{tikzcd}[sep=tiny]
  {}_{0}\mathrm{Mor}_{\mathbf{W}}
  \\
  \vdots
  \\
  {}_{n}\mathrm{Mor}_{\mathbf{W}}
\end{tikzcd}
\]
illustrates an $n$-category $\mathbf{W}$, then for $n$-categories $\mathbf{W}_{1}$ and $\mathbf{W}_{2}$ we can illustrate an $(n,n_{0})$-transfor for $n-n_{0}$ large enough as
\[
\begin{tikzcd}[row sep=tiny,column sep=large]
  {}_{0}\mathrm{Mor}_{\mathbf{W}_{1}}
  \arrow{rdd}{}
  &
  {}_{0}\mathrm{Mor}_{\mathbf{W}_{2}}
  \\
  \vdots
  &
  \vdots
  \\
  {}_{n_{0}}\mathrm{Mor}_{\mathbf{W}_{1}}
  \arrow{rdd}{}
  &
  {}_{n_{0}}\mathrm{Mor}_{\mathbf{W}_{2}}
  \\
  \vdots
  &
  \vdots
  \\
  \vdots
  &
  {}_{n_{0}+n_{0}}\mathrm{Mor}_{\mathbf{W}_{2}}
  \\
  \vdots
  &
  \vdots
  \\
  {}_{n-n_{0}}\mathrm{Mor}_{\mathbf{W}_{1}}
  \arrow{rdd}{}
  &
  \vdots
  \\
  \vdots
  &
  \vdots
  \\
  {}_{n}\mathrm{Mor}_{\mathbf{W}_{1}}
  &
  {}_{n}\mathrm{Mor}_{\mathbf{W}_{2}}
\end{tikzcd}
\]
The up to equivalencs part cannot be illustrated here in generality. The coherence laws are way to complicated for our means. But we saw a special case in subsection \ref{sec:nt} and we will see some more in subsubsection \ref{sec:hlm}.
\\
We now come to the categorification and oidification.
\begin{enumerate}
\item[(RC)]
Categorification is the following as of yet mostly informal process: for all $n$ when given a bunch of $n$-categories and a bunch of $(n,n_{0})$-transfors for all $0 \leq n_{0} \leq n$ subjected to some conditions only involving the $n$-categories and $(n,n_{0})$-transfors - this can be seen as a structure on these $n$-categories - then by replacing
\begin{description}
\item[Step 1]
each $n$-category by an $n+1$-category
\item[Step 2]
each $(n,n_{0})$-transfor by a suited $(n+1,n_{0})$-transfor
\end{description}
we have categorified the structure to $n+1$-categories. A more thorough discussion of this is given in \cite{66edf75b}. Let us emphasize what this means when categorifying a set: given a bunch of sets and a bunch of functions with both domain and codomain involving only these sets subjected to some conditions involving only these sets and functions then we can replace each set by a category and each function by a suited functor (the $(1,0)$-transfors). In particular, an equality of sets must be replaced by equivalence (what actually should have aready been done earlier) and equality of function by equivalence (the $(1,1)$-transfors up to coherent isomorphism).
\\
An example is to categorify a monoid $(M,\cdot,\mathrm{id})$. We replace $M$ by a category $\mathbf{C}$ and $\cdot$ by a functor from $\mathbf{C} \times \mathbf{C}$ to $\mathbf{C}$ as well as $\mathrm{id}$ by a functor from $\mathbf{1}_{M}$ to $\mathbf{C}$ which by the terminality of $\mathbf{1}_{M}$ in $\mathbf{Cat}$ is nothing but an object $1$ of $\mathbf{C}$. In the monoid properties (M1) and (M2) we replace $=$ by natural isomorphism. Then one has to look at the result what is missing to make it a sensible conception. We do this in subsubsection \ref{sec:hlm} and categorify a monoid to a monoidal category. An interesting peculiarity is that one can internalize monoids in monoidal categories. This is a further idea which is called microcosm principle coined by Baez and Dolan in \cite{0d7b89ad} where they give a definition of weak $n$-category and prove a version of the microcosm principle which says that when we categorify a structure we can internalize this structure there. Though as already mentioned weak $n$-categories are not used much anymore due to their addressed drawback, the paper \cite{0d7b89ad} is a seminal one on higher categories and does still provide good intuition. So we can still recommend reading it.
\item[(RO)]
Actually, we have already introduced the informal meta-idea of oidification on the quiet by two examples earlier in the notes. Namely the oidification of a monoid is a category while the oidification of a group is a groupoid. The process in these cases was that we had a structure on a set (monoid and group) and could show that it is equivalent to a certain kind of category (ordinary category and category with only isomorphisms, respectively) with exactly one object. Then we allowed the kind of the category to contain arbitrarily many objects to complete the oidification of the structure on a set. So oidification in higher category theory should be very much like categorification: take a structure on an $n$-category and transfer it to the next level. Hence oidification is described by the following meta-algorithm: Given some structure on an $n$-category for some $n \in \mathbb{N}$ then
\begin{description}
\item[Step 1]
show that the structure on this $n$-category is equivalent to a certain kind of $n+1$-category with precisely one $0$-morphism
\item[Step 2]
oidify by passing to this certain kind of $n+1$-category but allow arbitrarily many $0$-morphisms
\end{description}
\end{enumerate}
In the end, one might recognize that recursive categorification and oidification also allows to go from finite $n$ directly to $\infty$ after all. That it is presented recursively is arguably for historical reasons.
\\\\
All in all this subsection suggests that the $\infty$-version of structure is more fundamental than any finite version in the sense that the $\infty$-dimensional sets are the basic mathematical object rather than just ordinary sets.
\\
The following subsection applies many of the ideas sketched here in case of monoids yielding mostly formal definitions.
