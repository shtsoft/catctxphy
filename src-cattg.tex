This is the main - and by far the longest - chapter of these notes. It is all about category theory. However we will develop it in a set theory and not as first-order theory. Namely in Tarski-Grothendieck set theory. TG is (almost) ordinary mathematics as you might know it. The only difference is that we use {\glqq}Grothendieck universes{\grqq}. This is because in category theory one often has to do with large collections of things and Grothendieck universes are a neat way to handle these (way better than the classes in von-Neumann Bernays G\"{o}del (abbr. NBG) or even informal classes).\footnote{see appendix \ref{chap:tg} for a formulation of TG}
\\
On the one hand we pay special attention to homotopy theory and higher category theory albeit on a way more informal level than we do for ordinary category theory. Homotopy theory and higher category theory will play a role in almost any section of this chapter. But let us stress that it should be possible to understand a good part of this chapter without any prior knowledge of these theories (i.p. homotopy theory) as mentioned earlier.
\\
On the other hand we pay special attention to physics. Explicitly, we present an example serieses on so-called generalized spaces directly relating to physics. But also, implicitly, we have chosen the topics we think are most important to physics and still basic enough for an introduction.
\\\\
Let us now talk a bit about the sections of this chapter. Of course, we present the most basic ideas of category theory as you would find them in any text about general category theory but with certain emphases which are not so standard. Let us present this as a list:
\begin{enumerate}
\item[$\bullet$]
The trinity in section \ref{sec:trinity}: categories, functors, natural transformations. What is perhaps special in our notes is the early focus on higher categories which we already introduce in the trinity section \ref{sec:trinity}. We will then later see that this trinity is not so {\glqq}holy{\grqq} as one might guess from a first glance at category theory but rather an expression of the fact that $0$ and $1$ and $2$ are three natural numbers. 
\item[$\bullet$]
Duality in section \ref{sec:duality}. This is actually addressed in any serious text about category theory since a lack of understanding can make things a little bloated. Therefore we tried to emphasize this thorughout these notes as much as possible.
\item[$\bullet$]
Constructions on a category in section \ref{sec:constoncat}. We only discussed the ones important to us: product categories and comma categories. There are quite a few more interesting ones we do not discuss such as graphs. For this section there are certainly more complete sources.
\item[$\bullet$]
Universality in section \ref{sec:uni}. This is the most important thing about category theory which is why we extensively discuss it. We strived to be as intuitive as possible (what we actually try throughout the whole notes) with a special interest on parts which are particularly important to physics. We start from what {\glqq}universality{\grqq} intuitively means followed by the far reaching Yoneda lemma waiting there for us. It is not exaggerated to say that the Yoneda lemma dominates category in some sense. You will hopefully comprehend what we mean after a perusal of section \ref{sec:uni}. With limits and adjoints we will get to know the arguably most common {\glqq}universal constructions{\grqq}. We will prove most of the standard theorems for those. Then, a bit hidden in a subsubsection, we get to know a further universal construction - KAn extensions - which is also of major importance. Especially in homotopy theory they play a role.
\end{enumerate}
Well, this is what we consider the basics of category theory. But it is not where we stop. After the basics it is time to look a bit at the big picture of category theory from a more modern perspective. This is done in one section with a somewhat obscure title:
\begin{enumerate}
\item[$\bullet$]
Meta-ideas in section \ref{sec:metaidea}. Category theory and especially higher category theory is surrounded by many more or less informal ideas (therefore meta-ideas). We try to present these in a more or less informal style depending on how sophisticated the idea is and how much effort is acceptable for our purposes. In particular, you can find an informal introduction to higher category theory there with many references of interest.
\end{enumerate}
The last two sections are closely related to homotopy theory. And since homotopy theory might be important in physics we pay more attention to the subject than the ordinary introductory texts on category theory. Let us briefly discuss the topics we mean.
\begin{enumerate}
\item[$\bullet$]
Simplicial sets in section \ref{sec:sset}. In this section we discuss how simplicial sets play a major role in homotopy theory. In particular, we discuss how to use them in the classification of principal bundles by homotopy classes and cohomology ({\glqq}calculated{\grqq} with \v{C}ech cohomology).
\item[$\bullet$]
Fibrations in section \ref{sec:fibration}. For an algebraic topologist or even classical homotopy theorist it is obvious that this has to do with homotopy theory. We discuss a very general idea of fibrations and their duals (cofibrations) encompassing the notions from classical homotopy theory. The highlight is a pretty formal definition of {\glqq}stack{\grqq} which are essentially a homotopy-theortic version of sheaf.
\end{enumerate}
We hope we made it to give a reasonable overview of this chapter to save you from missing the forest for the trees.
