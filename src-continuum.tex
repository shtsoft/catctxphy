%\nocite{0b855cc5}
%\nocite{476fe2a3}
As humans we claim that we experience nature as a smooth continuum. So, as it proved very useful over the past centuries to describe nature mathematically, it seems that one needs a mathematical idea of what a smooth continuum is. Let us look at this idea a bit and see what it has to do with categoty theory.
\\
In Euclidean geometry continuity is built-in. It is a primitive concept. One does not explain what the continuity of a line is but one is expected to know from experience what it is. Now Euclidean style geometry is a bit limited as a theory to describe modern physics. Therefore, historically, Euclidean geometry was gradually replaced by mathematicians around 1900 with the nowadays mostly accepted mathematical theory of everything: some version of set theory. Set theory is all about collections of things and one has to model everything one wants to describe in terms of collections of things. In particular, a smooth continuum has to be modeled like that. One day someone came up with the idea of the well-known-but somehow magic set of real numbers $\mathbb{R}$ as a model for a continuum. This is the nowadays broadly accepted model for smooth continuum.
\\
But there is a perhaps severe problem with that model in (ZF-like) set theories with the choice axiom: the Banach-Tarski paradox. The paradox (on an intuitive level) is that one can decompose a unit ball in $\mathbb{R}^{3}$ into five disjoint sets and then compose these parts without any {\glqq}deformation{\grqq} into \underline{two} disjoint unit balls in $\mathbb{R}^{3}$. This seems a bit odd under the interpretation of smooth continuum. Even worse, some physical theories which in general yield quite good results struggle with the continuum (in that sense) as is the title and subject of the really worthwhile and easy to read notes \cite{476fe2a3}.
\\
One can now go ahead mitigating the struggles orthodoxly or take a step back exploring different ways. In the computer age a thinkable way is to give up the continuum altogether and go with a discrete model of nature. In fact, it is certainly in the bounds of possibility that nature is not a continuum at all. While a version of this finitistic point of view is pursued in \cite{0b855cc5} we can go in another direction taking the idea of built-in continuity up. This is where category theory comes into play: category theory provides an alternative to the tradtitional model $\mathbb{R}$ in the guise of topos theory\footnote{category theory subjected to some more axioms as we will see} providing a synthetic theory allowing to formally work with infinitesimals as physicist are used to do (informally) anyhow. Furthermore there is also so-called synthetic differential geometry which seems to be a worthwhile replacement of ordinary differential geometry. While we do not develop these ideas in these we provide the basis to understand these such that you can quickly pick it up from some other source. Note that \cite{wiki-nlab0000} lists some literature on the subject.
\\
Conclusively, for this paragraph, let us say that a proper notion of continuum is important when thinking about physics because it is an implicit assumption in physics to have one as we will see next. 
